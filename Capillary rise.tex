\documentclass{article}
\usepackage{tikz}
\usepackage{tkz-euclide}    
\usetikzlibrary{arrows.meta}

% ----------This code generates capillary rise of liquid in a capillary tube in contact with gas----------
% r: Capillary tube inner radius 
% \teta: Contact angle
% R=Radius of curvature for a semispherical G-L interface
% L: Liquid; 
% G: Gas; 
% P: Pressure; 
% T=Temperature; 



\begin{document}
\begin{tikzpicture}
	\begin{centering}
	
		%----------Drawing features----------
		% Outer box
		\draw [thick, black] (0,0) rectangle +(10,8);
		
		% G-L planner interface (in blue)
		\draw [pattern=north east lines, thick] (4.225,1) rectangle (4.3,6);
		\draw [pattern=north east lines, thick] (5.7,1) rectangle (5.775,6);
		\draw[color=blue, thick] (0,1.75) -- (4,1.75) arc (275:325:0.3cm); 
		\draw[color=blue, thick] (10,1.75) -- (6.01,1.75) arc (270:215:0.3cm); 
		\draw [blue, fill=blue, fill opacity=0.05] (0.6,2) -- (0.9,2)--(0.75,1.78) --cycle; % free-liquid level
		\filldraw[fill opacity=0.05,fill=blue, draw=none] (0,1.75) --(10,1.75) --(10,0)--(0,0); % liquid bulk
		
		% Curved interface in capillary tube (in blue)
		%\tkzDrawArc[thick,color=blue,name path=Cture](Ctr,A)(B) % interface
		\draw [draw=blue, thick] (4.3,4) arc (215:325:0.85cm) ; % confined liquid
		\fill [fill=blue, fill opacity=0.05,draw=none] (4.3,1.75) -- (4.3,4) arc (215:325:0.85cm) --(5.7,1.75) -- (4.3,1.75); % confined liquid
		
		% Radius of curvature and contact angle  (in red)
		\draw [red,fill=red] (5,4.66) circle (0.035cm); % center of curved interface
		\draw [red,-{Latex[length=2mm, width=1mm]}] (5,4.66) --(4.3,4) node [above, yshift=0.3cm, xshift=1cm, rotate=-40] {$R$}; % radius of curvature R (on right)
		\draw [red,-{Latex[length=2mm, width=1mm]}] (5,4.66) --(5.7,4); % radius of curvature (left)
		\draw [red, thin] (5.69,3.65) arc [start angle=275, end angle=210, radius=0.2cm]
		node [midway, below, xshift=-0.1cm] {$\theta$}; % contact angle
		\fill [red, fill=red, fill opacity=0.1] (5.7,4) -- (5.69,3.65) arc [start angle=275, end angle=210, radius=0.2cm] --(5.7,4);   % contact angle shade
		\draw [thin, red] (5.725,3.98) -- (5.0,3.2); % tangent line
		\draw [very thin, densely dashed, red] (5.725,3.98) --(6.6,5); % tangent line extension
		\draw [red, densely dashed] (4.3,4)--(5.7,4); % horizontal dashed line
		\draw [red, densely dashed] (5,4.66) -- (5,4); % vertical dashed line
		\draw [fill=red, fill opacity=0.1, red, rotate around={45:(5.7,4)}] (5.7,4) rectangle (5.6,4.1); %right angle
		\draw [fill=red, fill opacity=0.1, draw=red](5,4) rectangle (4.9,4.1); % right angle
		
		% ----------Labeling----------
		\draw[-{Latex[length=2mm, width=1mm]}] (3.5,5.5) node[left]{capillary tube} -- (3.8,5.5) --(4.2,5.2) ; % capillary tube
		\node [anchor=south, yshift=7cm,xshift=9cm] {$P_G$, $T$}; % pressure and temperture of gas
		\node [anchor=south, yshift=1cm,xshift=9cm,blue] {$P_L$, $T$}; % pressure and temperature of bulk liquid
		\draw[blue,{Latex[length=2mm, width=1mm]}-] (0.75,1.7) -- (0.85,1.3) -- (1.05,1.3) node[right, text width=2 cm, yshift=-0.19 cm]{G-L planner interface}; % planner interfacc for gas-liquid
		\draw [{Latex[length=2mm, width=1mm]}-{Latex[length=2mm, width=1mm]}] (4.3,5.8) -- (5.7,5.8) node [above, midway] {$2r$} ; % inner diameter of capillary tube
		
	\end{centering}
\end{tikzpicture}
\end{document}